\chapter{Introduction}
\label{Introduction}
\thispagestyle{empty}

\noindent Before the explosion of the Internet people used to get their movie recommendations from family and friends. Today it is as easy as signing up at Netflix and watching the first 5 minutes of a movie or series to get a recommendation. It was indeed Netflix that gave Recommender Systems (RS) the place it has nowadays in most of the products and services that we use. The Netflix Prize competition, a competition organized by Netflix in 2006 consisted in providing the best algorithm for estimating user ratings on various movies. RS is an information filtering technique \cite{zhang2019deep} that allows the prediction and ranking of items for users in terms of preference. A RS is running underneath many popular websites; Amazon uses it to power its purchase suggestions, Facebook for friend discovery, Twitter for personalized feed of tweets. All these service providers tap into the information a recommender system provides them in order to offer a customized user experience.

The main reasons for the proliferation of RS is the availability of user-item data created during the interaction with RS-powered products and user-/item-specific metadata. There two types of data power the two main paradigms of techniques in RS: collaborative filtering (CF) and content-based filtering (CBF). Both CF and CBF have been and still are widely used to build recommender systems mainly because of their simplicity, explainability and straightforward implementation. They both have pros and cons and because of this they are usually combined in a hybrid recommender system in order to get the best of both paradigms. 

In the context of filtering data for users, RS try to address two different but related problems -- the first one is predicting as correctly as possible how a specific user would rate a set of items; the second one is constructing a ranked list from a set of items for a specific user in decreasing order of preference. This second problem is called the Top-N recommendation problem due to the assumption that in most scenarios, users are mainly interested in the items that are ranked high by a RS.

Over the years many different algorithms have been adopted and created for the two RS problems introduced above. One class of algorithms is Matrix Factorization (MF). MF approaches are model-based; they construct a model from the user-item interaction matrix and use this model to predict ratings and order the set of items according to these predicted ratings to perform Top-N recommendations. MF models aim to represent users in a much lower dimensionality than that of the user-item interaction matrix by deriving latent factors for both items and users and predicting ratings from the multiplication of these factors. MF techniques in RS took off during the Netflix Prize competition and since then multiple derivatives have been developed.

The defining feature of MF is matrix multiplication which in itself is a linear operation. While powerful, this linearity can introduce some bottlenecks since the relation of users to items might be much more complex and only explained through some non-linear mapping. Such non-linearity can be easily achieved by using Multilayer Perceptrons (MLP) with non-linear activation functions. Deep Learning (DL), an extension of MLP-based Neural Networks to high number of hidden layers, introduced end-to-end training for Neural Networks and has been very successful in numerous tasks and fields like Computer Vision, Natural Language Processing, Signal Processing and recently also RS.

The success of DL has mainly been on discriminative modelling where a model is built to discriminate between samples and is able to assign new unseen data points the correct class. In 2014 Goodfellow et al. introduced Generative Adversarial Networks (GAN), a new framework built on top of Neural Networks for performing generative modelling. GAN have seen a great interest in the ML community for their ability to generate images from noise with a higher resolution than previous approaches. They do this by approximating the distribution of the training data and are able to sample from this learned distribution.

However, despite their clear presence in ML research, GANs have not been widely used in RS. In this thesis we show a GAN based recommender systems. \hl{This new approach is based on the MF model where instead of the linear matrix multiplication operation???}, we utilize the adversarial learning nature of GANs to learn the latent factors. Our work is based on the assumption that the representation in latent factors that MF builds for users/items can be taken as the preference of users on the latent factors and by analogy how much of these factors are present in each item. Such preference can be quite complex or even degenerate and GANs are suited to estimate such distributions. We give the formulation and architecture of our approach and evaluate it across multiple datasets which are commonly used to test new RS algorithms.