\newpage
\chapter*{Abstract}

\addcontentsline{toc}{chapter}{Abstract}

The last decade has seen an exponential increase in the amount of available information thanks to the ever-growing number of connected devices and interaction of users with online content like social media, e-commerce, etc. While this translates in more choices for people given their diverse set of preferences, it makes it difficult for them to explore this vast amount of information. Recommender systems (RS) aim to alleviate this problem by filtering the content offered to users by predicting either the rating of items by users or the propensity of users to like specific items. The latter is known as Top-N recommendation in the RS community and it refers to the problem of recommending items to users, preferably in the order from most likely-to-interact to least likely-to-interact.

RS use two main approaches for providing recommendations to users; collaborative filtering and content-based filtering. A third approach, hybrid RS, can be constructed from a combination of the two. One of the main algorithms used in collaborative filtering is matrix factorization (MF) which constitutes in estimating the user preferences by decomposing a user-item interaction matrix into matrices of lower dimensionality of latent features of users and items \cite{koren2009matrix}.

The burst of massive data has triggered a corresponding response in the machine learning community in trying to come up with new techniques to extract relevant information from data. One such technique is Generative Adversarial Nets \cite{goodfellow2014generative} (GAN) proposed in 2014 by Goodfellow et al. which initiated a fresh interest in generative modelling. Despite their popularity, GANs and more generally adversarial training have not been widely applied in RS.
In this thesis we investigate a novel approach that estimates the user and item latent factors in a matrix factorization setting through the application of Generative Adversarial Networks for generic Top-N recommendation problem. We detail the formulation of this approach and show its performance through different experiments on well know datasets in the RS community, with both implicit and explicit feedback data.